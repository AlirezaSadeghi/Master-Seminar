\section{روش ارائه شده}\label{proposed}
یکی از بهترین روش‌های ارائه شده برای توصیف ویدئو، پژوهش 
\cite{Yu2016}
است که در بخش 
\ref{hierarchical-rnns}
به آن اشاره شد. در این روش از سلسله‌مراتبی از شبکه‌های بازگردنده در کنار توجه و یادگیری چندحالته استفاده می‌شود. یکی از نقاط ضعف این روش اما استفاده‌ی حداقلی از ویژگی‌های تصویری ویدئو است. ویدئو‌ها به عنوان یک منبع تصویری زمانی-مکانی، حاوی اطلاعات بسیاری هستند که استفاده‌ی مناسب از آنها می‌تواند به افزایش دقت قابل توجه مدل منجر شود.  در این روش، از خروجی لایه‌ی بازگردنده‌ی ایجاد‌کننده جمله، برای به دست آوردن ویژگی‌های مهم تصویر (توجه روی ویژگی‌های تصویری) استفاده می‌شود و سپس با به دست آمدن ویژگی‌های تصویری مرتبط با جمله‌ی درحال ایجاد، این ویژگی‌ها با دانش‌زبانی مدل در لایه‌ی چند‌حالته ترکیب می‌شوند تا کلمه‌ی بعدی خروجی را ایجاد کنند.
روش پیشنهادی ما در این گزارش، بر مبنای پژوهش \cite{Yu2016} است. برای رفع مشکل اشاره شده، ما یک چرخه‌ی بازخورد از خروجی مدل با توجه به ویژگی‌های تصویری به شبکه‌ی ایجاد کننده‌ی جمله وارد می‌کنیم. بدین صورت سعی می‌کنیم تا مدل در زمان آموزش، یادگیرد تا از ویژگی‌های تصویری به دست آمده در ایجاد کلمات بعدی استفاده کند (تنها از کلمات ایجاد شده تا به حال برای ایجاد کلمه‌ی بعدی استفاده نکنیم). در واقع روش‌پیشنهادی اولیه‌ی ما بر پایه‌ی افزایش در هم تنیدگی ویژگی‌های تصویری و متنی بنا شده است.
برای بررسی این مدعا،  ساختار  ارائه شده در \cite{Yu2016} را بدین نحو تغییر می‌دهیم که خروجی لایه‌ی چند‌حالته به ورودی شبکه‌ی بازگشتی ایجاد‌کننده‌ی جمله وارد می‌شود. این تغییر باعث افزایش ابعاد شبکه‌های ایجاد‌کننده‌ی جمله، پاراگراف و تعبیه‌ی آخرین جمله می‌شود. این افزایش ابعاد متعاقباً منجر به افزایش پارامتر‌های لازم برای آموزش شبکه می‌شود که برای جلوگیری از بیش‌برازش، از روش 
\LR{Dropout}
استفاده می‌کنیم در کنار بهره‌گیری از منظم‌ساز‌های \LR{L1} و \LR{L2} استفاده می‌کنیم. در نهایت معماری پیشنهادی به صورت زیر است:

\begin{figure}[ht!]
	\centering
	\includegraphics[width=170mm]{images/proposed.jpg}
	\caption{نمونه‌های معماری ارائه‌شده برای توصیف ویدئو}
\end{figure}


