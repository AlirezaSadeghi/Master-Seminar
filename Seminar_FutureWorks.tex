\section{کارهای آتی}\label{future_works}
یکی از روش‌هایی که اخیرا در زمینه‌ی پردازش ویدئو مورد استفاده قرار گرفته و منجر به بهبود‌های قابل توجهی در دقت این روش‌ها شده است، استفاده از شبکه‌های بازگردنده‌ی دو طرفه
\LTRfootnote{Bi-directional LSTM, BiLSTM}
 است. در این شبکه‌ها، علاوه بر اینکه از فریم‌های ویدئو به ترتیب زمانی برای آموزش شبکه استفاده می‌شود، از فریم‌های آینده‌ی ویدئو نیز برای آموزش یک شبکه‌ی بازگردنده‌ی در جهت مخالف شبکه‌ی اولیه استفاده می‌شود. در واقع هدف این دست از روش‌ها، استفاده از رخداد‌های آینده‌ی ویدئو در ایجاد توصیف برای ویدئو است. از کارهای آینده‌ی این پژوهش، بررسی نحوه‌ی ادغام این دست از شبکه‌ها در معماری مد‌نظر و آموزش‌دادن آنها است.

همچنین یکی از نقاط ضعف شبکه‌های بازگردنده‌، عدم توانایی در مدل‌کردن دنباله‌های طولانی است. علت رخداد این امر نیز وجود تنها یک‌ بردار حالت در این شبکه‌ها است. اخیرا یک دسته از شبکه‌های عصبی به نام شبکه‌های حافظه
\LTRfootnote{Memory‌‌ Networks}
\cite{weston2014memory}
است که هدف آنها، پاسخ‌دهی به وابستگی‌های زمانی پیچیده و با فاصله‌ی زیاد در داده‌ها است. یکی دیگر از کارهای آتی این پژوهش، بررسی به‌کارگیری این شبکه‌ها در کنار شبکه‌های بازگردنده‌ی موجود برای ایجاد جملات زبان‌طبیعی است.

در نهایت، خلاصه‌ای از مراحل و میزان پیشرفت این پژوهش در جدول \ref{tab:Timing} آمده است. 

 \begin{table}[h!]
	\caption{جدول زمان‌بندی\label{tab:Timing}}
	\begin{center}
		\begin{tabular}{|r|c|c|c|}
			\hline
			\rl{عنوان فعالیت}&\rl{مدت زمان لازم}&\rl{درصد پیشرفت}&\rl{زمان اتمام}\\ \hline \hline
			\rl{مطالعه و بررسی روش‌های موجود و راه‌کارهای قابل استفاده  }&\rl{3 ماه}&100&\rl{شهریور ۹۴}\\ \hline
			\rl{آزمایش روش‌های موجود مقایسه آن‌ها}& \rl{۲ ماه}&100&\rl{آبان  ۹۴}\\ \hline
			\rl{بررسی و یافتن کاستی‌های روش‌های موجود}&\rl{۱ ماه}&75&\rl{آبان ۹۴}\\ \hline
			\rl{ پیشنهاد و پیاده‌سازی و ارزیابی روش جدید}&\rl{۴ ماه}& 5&\rl{اسفند ۹۴}\\ \hline
			\rl{ارزیابی روش نهایی و مقایسه با روش‌های دیگر}&\rl{۲ ماه}&0&\rl{اردیبهشت ۹۵}\\ \hline
			\rl{نگارش پایان‌نامه}&\rl{۲ ماه}&0&\rl{تیر ۹۵}\\ \hline
		\end{tabular}
	\end{center}
\end{table}